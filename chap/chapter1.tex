Das nachfolgende Dokumente enthält Erläuterungen und Details zur LaTeX-Vorlage 
für Praktikumsprotokolle. Es handelt sich dabei um Protokolle für die 
physikalischen Anfängerpraktika der Fakultät für Physik des KIT. Das Dokument 
richtet sich an Studierende des KIT, die am Praktikum teilnehmen. Die Vorlage 
wird von der Fachschaft für Physik bereit gestellt.

Um ein Protokoll gemäß der üblichen Formatierungsrichtlinien erstellen zu 
können sind zwei Dinge vorausgesetzt. Zum Einen muss der Verfasser sich mit den 
Richtlinien auskennen. Empfehlenswert sind hierbei der Kurs zum 
wissenschaftlichen Schreiben vom House of Competence sowie eine kurze 
Zusammenfassung aller Formatierungsregeln von Herrn Seitz-Moskaliuk, die auf 
der Seite des Praktikums zu finden sind.

\hfill\href{
http://www-ekp.physik.uni-karlsruhe.de/~simonis/praktikum/allgemeines/Protokoll-
Formatierungshinweise.pdf}{\nolinkurl{
http://www-ekp.physik.uni-karlsruhe.de/~simonis/praktikum/allgemeines/ 
Protokoll-Formatierungshinweise.pdf}}

Die Richtlinien selbst sollen aber nicht Thema dieses Dokuments sein. Zum 
Anderen muss der Verfasser wissen, wie er diese Richtlinien mit der verwendeten 
Software umsetzen kann. Im Falle von LaTeX ist das Umsetzen dieser Richtlinien 
meist enorm einfach, nur der Einstieg in die Verwendung von LaTeX gestaltet 
sich für viele Studenten nicht so leicht.

Um den Studenten diesen Einstieg so leicht wie möglich zu machen, hat sich die 
Fachschaft für Physik dazu entschieden, diese Vorlage zur Verfügung zu stellen.

Diese Vorlage setzt die Verwendung des Textsatzsystems LaTeX voraus. Wer sich 
gegen LaTeX entscheidet, braucht ab dieser Stelle nicht weiter zu lesen. Das 
Erstellen des Praktikumsprotokolls ohne LaTeX ist durchaus möglich. Mehr Details 
dazu im nächsten Kapitel.

Zu der Vorlage gibt es ein Beispielprotokoll, welches mithilfe der Vorlage 
erstellt wurde. Es ist zu beachten, dass diese Vorlage nur ein Beispiel für 
Textsatz und Formatierungen ist – auf ein formal und inhaltlich korrektes 
Protokoll wurde nicht geachtet. Musterprotokolle dieser Art können auf den 
Seiten des Anfängerpraktikums der Fakultät gefunden werden.
