Das nachfolgende Dokumente enthält Erläuterungen und Details zur LaTeX-Vorlage für Praktikumsprotokolle sowie einige allgemeine Ratschläge zum technischen Verfassen von Praktikumsprotokollen. Es handelt sich dabei um Protokolle für die physikalischen Anfängerpraktika der Fakultät für Physik des KIT. Das Dokument richtet sich an Studierende des KIT, die am Praktikum teilnehmen. Die Vorlage wird von der Fachschaft für Physik bereit gestellt.

Wissenschaftliche Texte sollten einige Anforderungen erfüllen. Beim Einhalten dieser Anforderungen stellen sich dem Studenten zwei wesentliche Probleme. Zum Ersten muss er die nötige Kenntnis darüber besitzen, welche Anforderungen wie zu erfüllen sind. Er muss also die formellen Regeln kennen, die beim wissenschaftlichen Schreiben einzuhalten sind. Auf dieses Problem wird hier nicht eingegangen. Entsprechende Kurse des HoC zum wissenschaftlichen Schreiben und Präsentieren sind für das Erlernen dieser Formalien empfehlenswert. Zum Zweiten jedoch muss der Student wissen, wie diese Regeln technisch, d.h.~beim Erstellen eines Dokuments am Computer, umzusetzen sind. Um die Studenten bei dem Umgang mit diesem Problem zu unterstützen, steht diese LaTeX-Vorlage zur Verfügung.

Es soll kurz angemerkt werden, dass das Erstellen von Abbildungen und Tabellen, das setzen von Formeln und Einheiten sowie weitere äußere Merkmale einer wissenschaftlichen Arbeit grundsätzlich gewissen Richtlinien unterliegen, die in Form von DIN- und ISO-Normen festgelegt sind. Diese Normen sollen hier keineswegs erläutert werden. Um diese im Detail zu kennen, ist auch hier der Kurs zu wissenschaftlichem Schreiben des HoC zu empfehlen, der solche Feinheiten des Verfassens wissenschaftlicher Dokument thematisiert. Einführung in die Grundlagen und kleinere Hilfestellungen erhält man von den Tutoren des Anfängerpraktikums. Die Umsetzung der Richtlinien fällt mit LaTeX in vielen Fällen sehr leicht. Im letzten Kapitel dieser Anleitung finden sich ein paar Tipps direkt zusammen mit den Befehlsbeschreibungen für das Erstellen von Abbildungen, Tabellen und Einheiten. Es sei weiter angemerkt, dass man beim späteren Arbeiten an einem Institut/in einer Forschungsgruppe darauf achten sollte, welche Richtlinien verlangt werden. Diese können von Gruppe zu Gruppe in einem Fachbereich oder an einer Universität variieren. Die speziellen Richtlinien können dabei die offiziellen Richtlinien erweitern oder in Spezialfällen sogar gegen diese verstoßen. Es empfiehlt sich daher beim späteren Schreiben einer Arbeit immer stets bei der eigenen Gruppe/dem eigenen Institut nach zu fragen, welche Richtlinien verlangt werden.

Diese Vorlage setzt die Verwendung des Textsatzsystems LaTeX voraus. Wer sich gegen LaTeX entscheidet, braucht ab dieser Stelle nicht weiter zu lesen. Das Erstellen des Praktikumsprotokolls ohne LaTeX ist durchaus möglich. Mehr Details dazu im nächsten Kapitel.

Zu der Vorlage gibt es ein Beispielprotokoll, welches mithilfe der Vorlage erstellt wurde. Es ist zu beachten, dass diese Vorlage nur ein Beispiel für Textsatz und Formatierungen ist – auf ein formal und inhaltlich korrektes Protokoll wurde nicht geachtet. Musterprotokolle dieser Art können auf den Seiten des Anfängerpraktikums der Fakultät gefunden werden.